\documentclass{article}
\usepackage[utf8]{inputenc}

\title{crisis de los fundamentos y la computacion}
\author{pablo.davila }
\date{Marzo 2020}
\usepackage[spanish]{babel}
\begin{document}

\maketitle

\section{Introduction}
cuantos de nosotros hemos pensado que las matemáticas son infalibles y que pueden dar respuesta a cualquier problema?, a finales del siglo XIX la mayoría de los matemáticos estaban convencidos de que en efecto las matemáticas eran infalibles, pero, ¿realmente los son?.
En el siguiente ensayo veremos como el análisis de conjuntos matemáticos y el matemático Kurt Godel, culminaron en la demostración de que las matemáticas no son infalibles yendo en contra de la mayoría de matemáticos y desmotando el llamado Programa Formalista, y como esto conllevo al nacimiento de la computación y algoritmos. 
\section{desarollo}
En lo personal, si en algún momento me hubiese planteado de manera seria pensar en cuales fueron los causantes de lo que hoy conocemos como computación, nunca hubiera llegado a la conclusión de que las paradojas fueron las principales culpables. Al parecer, una paradoja poco tiene que ver con  las matemáticas, y menos con la computación, sin embargo este juego de palabras que muchas veces no parecen tener mucho sentido (las paradojas) termino desatando un conflicto grandísimo entre los matemáticos, y en tratar de responder a la pregunta ¿son las matemáticas capaces de resolver todo, y si no, cuales son sus limites?, preguntas que culminaron en la denominada crisis de los fundamentos. Por ejemplo una paradoja puede ser “esta frase es falsa” , si es falsa entonces no es falsa y si no es falsa entonces en verdadera, por tanto es falsa, estas frases para la mayoría pueden ser trucos insignificantes que no hay prestarle especial atención, sin embargo muchos matemáticos se lo tomaron muy enserio, pero, ¿que tiene que ver esto con las matemáticas?, bueno, podríamos hablar de paradojas de una manera mas matemática, como por ejemplo pensar en un conjunto que contenga todos los conjuntos que no son elementos de si mismo, ¿es este conjunto elemento de si mismo?, bueno es casi igual a preguntarse si la frase “esta frase es falsa” es falsa o no. Con respecto a la solución de este grandísimo problema el matemático David Hilbert con su filosofía de “debemos saber y sabremos” propuso que por medio de axiomas o postulados se podría llegar a cualquier demostración, lo que se denomino “Programa Formalista”, el cual nos decía que por medio de 3 reglas fundamentales podría las matemáticas llegar a cualquier solución, la primera regla era que el problema debía ser consistente, es decir que solo podía ser falso o verdadero, no ambos, no eran aceptadas las ambigüedades; la segunda regla consistía en que las soluciones debían ser completas, es decir que por medio de axiomas se pueda llegar a los movimientos o demostraciones necesarias para la solución del problema, y por ultimo se tenia que llegar a la solución con movimientos finitos; estas reglas parecían tener muchísimo sentido, por lo cual muchas personas en el ámbito de las matemáticas se unieron al nombrado Programa Formalista, sin embargo la aceptación del programa formalista no duro por mucho tiempo, pues el matemático Kurt Godel demostró que una solución con finitos posibles “movimientos” no era consistente y completa a la vez concluyendo que las matemáticas en efecto no eran infalibles, he aquí la muerte del Programa Formalista. Pero ¿que es un axioma?, de una manera extremadamente simplificada un postulado o axioma es una afirmación matemática que no puede ser demostrada con calculos, sin embargo una afirmación en la que todos creemos, por ejemplo : por dos puntos pasa una recta, esta afirmación es un postulado o axioma, no se puede demostrar con matemáticas rigurosamente hablando, sin embargo todos lo damos por cierto y de estos axiomas nacen todos los teoremas conocidos. Sin embargo esta idea de que por medio de postulados se podría llegar a cualquier demostración como ya sabemos fue un completo fracaso, por el contrario fue ampliamente aceptada la idea de que podríamos saber cuando un problemas tiene solución, o no la tiene, a raíz de esto se dio por primera vez la idea de la computación, la idea de que este uso de axiomas para saber si un problema es soluble o no podría ser mecanizado.
En un principio este conocimiento de la mecanización de soluciones a problemas hizo surgir una pregunta muy importante al razonamiento de Alan Turing, esta pregunta fue: “¿que le seria imposible a semejante maquina?”, y Alan Turing casi de inmediato a hacer publica esta pregunta encuentra el mismo la respuesta al llamado algoritmo imposible, imaginemos un algoritmo que entes de iniciar la solución programada del problema supiera si va poder solucionarlo, o si se quedara en un bucle infinito o algún tipo de acontecimiento que le impidiera solucionar el problema, es este el algoritmo imposible, y a raíz de este, surgió otro concepto mas en el ámbito de la computación, dicho concepto se nombro en un principio como la “entropia de los programas”, mas tarde el mundo conoció este concepto como la complejidad computacional. 

\section{conclusiones}
• la crisis de los fundamentos pese a ser visto como algo negativo en el momento en el que surgió, fue la responsable del nacimiento de la computación.
•  La computación surge en un principio como la mecanización de los axiomas para dar soluciones a problemas matemáticos, sin embargo dicha mecanización de axiomas paso por una rápida evolución convirtiéndose de esta manera en parte fundamental de nuestras vidas, dicha mecanización nos llevo al espacio, a entender poco a poco el cosmos y a curar enfermedades que parecían incurables. \cite{NormaIPC} \cite{NormaIPA} \cite{NormaIPR}

\bibliographystyle{plain}
\bibliography{bibliografia.bib}
\end{document}
